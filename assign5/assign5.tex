\documentclass[answers]{exam}
\usepackage{listings, lstautogobble}
\usepackage{color}
\usepackage{cancel}
\usepackage{amsmath}
\usepackage{amssymb}
\usepackage{qtree}
\usepackage{tabu}

\renewcommand{\O}[1]{\mathcal{O}\left(#1\right)}
\allowdisplaybreaks

\begin{document}
    \shadedsolutions
    \lstset{autogobble=true, commentstyle=\ttfamily}
    \definecolor{SolutionColor}{rgb}{0.95,0.95,0.95}

    \begin{center}
        \textbf{COL100 Assignment 5}\\
        Arpit Saxena\\
        2018MT10742\\
        Group 29
    \end{center}

    \begin{questions}
        \question Derive the proof that in the average case, the running time complexity of quick sort is \(\O{n \log n}\)
        \begin{solution}
            For deriving the time complexity, we'll refer to the following implementation for quick sort:
            \begin{lstlisting}[language=Python]
                def partition(a, left, right):
                    elem = a[right]
                    i = left
                    j = right
                    #INV: a[left..i-1] < elem <= a[j+1..right], left <= pos <= right
                    while i <= j:
                        if a[j] >= elem:
                            j -= 1
                        else:
                            a[i], a[j] = a[j], a[i]
                            i += 1

                    a[j + 1], a[right] = a[right], a[j + 1]
                    pos = j + 1

                    #assert: a[left..pos-1] < a[pos] <= a[pos+1..right], 
                    # left <= pos <= right
                    return pos

                def qsort(a, left, right):
                    #assert: a is an array

                    if left < right:
                        pos = partition(a, left, right) 
                            #assert: partitioned with right element as pivot.
                            #pos indicates the position of pivot in partitioned array
                            #   a[left..pos-1] < a[pos] <= a[pos+1..right], 
                            # left <= pos <= right
                        qsort(a, left, pos - 1)
                        qsort(a, pos + 1, right)
                    
                    #assert: a[left..right] is sorted

                def quicksort(a):
                    qsort(a, 0, len(a) - 1)
                    #assert: a[0..len(a)-1] is sorted
                    return a
            \end{lstlisting}

            We define \(n = right - left + 1\). Clearly, the average time complexity of
            quick sort only depends on \(n\).

            First, we look at the partition function. We observe that the while loop runs exactly
            \(n\) times irrespective of the contents of a.\\
             \(\therefore \text{Partition function takes } \O{n} \) time.

            Let \(T(n)\) denote the time complexity of the qsort function.
            Then, we have the following recurrence:
            \begin{equation*}
                T(n) =
                \begin{cases}
                    0 \text{ if } n = 1 \\
                    \O{n} + T(p) + T(n - p - 1) \text{ otherwise}
                \end{cases}
            \end{equation*}
            Here, \(p = pos - left\) denotes the size of the left partitioned array and
            thus, \(n - p - 1\) denotes the size of the right partitioned array.

            We note that sizes of the partitioned arrays would depend on where the element we are
            partitioning with respect to ends up at.
            On random data, there is equal chance for the partition to be at any place in the array.
            There, on average
            \begin{equation*}
                T(n) =
                \begin{cases}
                    0 \text{ if } n = 1 \\
                    \O{n} + \dfrac{1}{n}\displaystyle
                    \sum_{p=0}^{n-1}{T(p) + T(n - p - 1)} \text{ otherwise}
                \end{cases}
            \end{equation*}

            For \(n > 1\),
            \begin{equation*}
                \begin{aligned}
                    T(n) &= cn + \dfrac{1}{n}\sum_{p=0}^{n-1}{T(p) + T(n - p - 1)} \\
                         &= cn + \dfrac{2}{n}\sum_{p=0}^{n-1}{T(p)} \\    
                    nT(n) &= cn^2 + 2\sum_{p=0}{n - 1}{T(p)} \\
                    (n - 1)T(n - 1) &= c(n - 1)^2 + 2\sum_{p=0}{n - 2}{T(p)} \\
                    nT(n) - (n-1)T(n-1) &= c(2n - 1) + 2T(n - 1) \\
                    \text{ Dropping the } &-c \text{ as it is a constant and won't have any effect on the order} \\   
                    nT(n) - (n-1)T(n-1) &= 2cn + 2T(n - 1) \\
                    nT(n) - (n + 1)T(n - 1) &= 2cn \\
                \end{aligned}
            \end{equation*}

            \begin{equation*}
                \begin{aligned}
                    \frac{T(n)}{n+1} - \frac{T(n-1)}{n} &= \frac{2c}{n + 1} \\
                    \frac{T(n)}{n} - \frac{T(n-2)}{n-1} &= \frac{2c}{n} \\
                    \vdots \\
                    \frac{T(2)}{3} - \frac{\cancelto{0}{T(1)}}{2} &= \frac{2c}{2} \\
                \end{aligned}
            \end{equation*}

            Adding the equations, we get
            \[\frac{T(n)}{n + 1} = 2c\left\{\frac{1}{2} + \frac{1}{3} + \cdots + \frac{1}{n + 1}\right\}
                \leq 2c\int_{1}^{n}{\frac{1}{x} \text{d}x} = 2c\ln{n}
            \]
            Note that the inequality can be easily shown from the graph of \(y = \frac{1}{x}\) and
            taking the area under it.

            \(\therefore\) In the average case, \(T(n) = \O{n \log{n}}\)

            The taken by the quicksort function would thus be equal to \(T(n)\)
            where \(n = len(a)\)

            \(\therefore\) Time complexity of quicksort in the average case is equal to
            \(\O{n \log n}\) where \(n\) is equal to the number of elements to be sorted.
            \(\blacksquare\)
        \end{solution}
    \end{questions}
\end{document}